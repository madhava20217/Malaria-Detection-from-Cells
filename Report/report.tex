\documentclass[10pt,twocolumn,letterpaper]{article}

\usepackage{cvpr}
\usepackage{times}
\usepackage{epsfig}
\usepackage{graphicx}
\usepackage{amsmath}
\usepackage{amssymb}
\usepackage[normalem]{ulem}
\useunder{\uline}{\ul}{}
% Include other packages here, before hyperref.

% If you comment hyperref and then uncomment it, you should delete
% egpaper.aux before re-running latex.  (Or just hit 'q' on the first latex
% run, let it finish, and you should be clear).
\usepackage[breaklinks=true,bookmarks=false]{hyperref}

\cvprfinalcopy % *** Uncomment this line for the final submission

\def\cvprPaperID{****} % *** Enter the CVPR Paper ID here
\def\httilde{\mbox{\tt\raisebox{-.5ex}{\symbol{126}}}}

% Pages are numbered in submission mode, and unnumbered in camera-ready
%\ifcvprfinal\pagestyle{empty}\fi
\setcounter{page}{1}
\begin{document}

%%%%%%%%% TITLE
\title{Malaria Detection using Machine Learning}

\author{Harshit Goyal\\
{\tt\small harshit20203@iiitd.ac.in}
% For a paper whose authors are all at the same institution,
% omit the following lines up until the closing ``}''.
% Additional authors and addresses can be added with ``\and'',
% just like the second author.
% To save space, use either the email address or home page, not both
\and
Madhava Krishna\\
{\tt\small madhava20217@iiitd.ac.in}
\and
Shreya Bhatia\\
{\tt\small shreya20542@iiitd.ac.in}
\and
Srishti Singh\\
{\tt\small srishti20409@iiitd.ac.in}
}

\maketitle
%\thispagestyle{empty}

%%%%%%%%% ABSTRACT
\begin{abstract}
   Malaria is a life-threatening spread by infected Anopheles mosquito bites. Existing means of diagnosis include light microscopy and rapid diagnostic tests, which are used in conjuction to provide accurate results. However, the costs associated with them, in terms of human capital and time required, are immense.
   
   We seek to provide a complementing approach to infection classification using machine learning, which is fast and inexpensive. By training different algorithms like logistic regression, boosted decision trees, support vector machines and convolutional neural networks on images of varying sizes and using image transformations to augment the dataset, we conduct a comprehensive study on model accuracy and inferencing time.
\end{abstract}

%%%%%%%%% BODY TEXT
\section{Introduction}

Malaria is an infectious disease caused by 5 species of the Plasmodium parasite: \textit{Plasmodium falciparum}, \textit{Plasmodium vivax}, \textit{Plasmodium malariae}, \textit{Plasmodium ovale} and \textit{Plasmodium knowlesi}, spread by bites of the Anopheles mosquito. An estimated 241 million infections and 627,000 deaths occurred in 2020-21 \cite{worldmalaria}. 

\subsection{Testing Methods}
The infection can be detected using microscopy tests, Rapid Diagnostic Tests (RDTs) and serological tests.\cite{cdcmalaria}

Microscopy tests involve collecting and dyeing a thin or thick blood specimen with Giemsa or Wright's stain to detect infections visually and ascertain the percentage of infected to uninfected cells.

RDTs indicate whether the patient is infected with one of the species of the malaria-causing \textit{Plasmodium} and provide results in about 15 minutes. However, they fail to indicate a premature infection and negative RDT results need further evaluation. Using microscopy is also advised with positive results, so that the proportion of parasitized to uninfected cells can be determined.

Serological tests examine whether antibodies for the infection are present. They are mostly used for screening blood donors, testing for questionable diagnosis accompanied with treatment.

\subsection{Role of Machine Learning}
Numerous machine learning models have been proposed which segment a Whole Slide Image to identify red blood cells (RBCs) and classify these RBCs with a secondary trained model using deep neural network architectures and boosted trees. Our goal is to provide a computationally modest model with a good accuracy for the latter task, and provide meaningful results on how image dimensions and colour channel affect the accuracy of the above proposed models. Once these goals have been achieved, we will delve into image segmentation techniques to isolate RBCs from wholeslide images to pipeline the whole inferencing, if time permits.

\section{Literature Survey}
%
Poostchi \etal created datasets, processed them, and tested a variety of algorithms like Naive Bayes, Logistic Regression, Decision Tree, Adaboost, SVMs, Neural Networks and Deep Neural Networks (DNNs). They also considered deployment of ML-based systems to diagnose Malaria \cite{POOSTCHI201836}.

Liang \etal. compared a 16-layer Convolutional Neural Network (CNN) model with transfer learning for classifying single infected cells. They noted that the CNN achieved greater accuracy, sensitivity, specificity, f1-score and Matthew's correlation coefficient over the latter \cite{7822567}.

Pan \etal explores preprocessing of images and segmentation to isolate single cells from wholeslide images. They used encoders and discussed encoder architectures for feature extraction, and discussed spatial and feature-space interpolation for enriching the dataset. They consistently noted higher performance of models trained on augmented datasets \cite{Pan18}.

Fuhad \etal implemented a CNN-based model and implemented data augmentation techniques like random rotations, zoom, translations, shear and horizontal flips. They used CNNs as an autoencoder to extract and reduce features and used SVM and KNN algorithms to classify the images. They conducted knowledge distillation in order to prune the trained model and reduce its complexity, and deployed the resulting commpressed model to mobile and web-based applications. They also conducted analyses if common mobile phones could utilize the model to classify cells.\cite{fuhadmalaria}.

\section{Dataset}
The dataset used was publicly available, courtesy of  images were taken at Chittagong Medical College Hospital, Bangladesh.\cite{datasetref}

\subsection{Dataset Description}
The dataset contains 13,799 parasitized and 13,799 uninfected image samples containing 3 colour dimensions for a total of 27,588 images. The images are of varying sizes. The maximum height and width was 385 and 394 pixels respectively. The minimum height are width was 40 and 46 pixels respectively. The mean height and width was 133 and 132 pixels respectively. The median height and width was 130 pixels. The mean aspect ratio of the images is 1.0138.

Out of the 27,588 images, 647 parasitized and 750 unparasitized images were misclassified \cite{fuhadmalaria}.

\begin{figure}[t]
   \begin{center}
      \includegraphics[width=1\linewidth]{../Plots/image_vis.png}
   \end{center}
      \caption{True and false parasitized and uninfected images.}
   \label{fig:malaria_image}
\end{figure}

\section{Methodology}

In order to reduce redundancy and obtain results in the form we desired, we focused on creating modules with specific objectives. Modules for downloading and setting up the dataset, for labelling the said dataset, to perform evaluation of models and transform images to augment the dataset were created. We have included the package requirements in a requirements.txt file, which can be used for easy installation using pip.

\subsection{Exploratory Data Analysis}
In order to determine which colour channel was the clearest with respect to the identification of the chromatin dot characteristic to the parasitized cell, we plotted the images in different colour channels and in grayscale.

Out of the plotted images, the green channel showed the maximum isolation of the chromatin dot. We also visualised inverted images and noticed that the green channel had isolated the chromatin dot the most.
\begin{figure}[t]
   \begin{center}
      \includegraphics[width=1\linewidth]{../Plots/Comparison between channels.png}
   \end{center}
      \caption{Comparison between various colour channels for a true parasitized cell.}
   \label{fig:channel_comparison}
\end{figure}

We experimented with colour model transformations, and noticed that some models applying non-linear transformations (like HSV, HLS) captured the chromatin dot in parasitized cells better.

\begin{figure}[t]
   \begin{center}
      \includegraphics[width=1\linewidth]{../Plots/hsv_conversion.png}
   \end{center}
      \caption{Conversion to HSV space from RGB.}
   \label{fig:HSV_space}
\end{figure}

\subsubsection{Cluster Visualisation}
To visualise them, the images were first resized to a 50x50 colour format, each pixel value rescaled by 1/255, and each image finally flattened to a 50*50*3 length array. We used Euclidean distance as the distance metric and used the t-SNE algorithm to reduce dimensions to 2. We also used KMeans clustering to determine similarity clusters, but there was no clear relation between the natural clusters and the clusters output by K-Means (figure 5).

\begin{figure}[t]
   \begin{center}
      \includegraphics[width=1\linewidth]{../Plots/kmeans_natural labels.png}
   \end{center}
      \caption{KMeans and Natural Labels. The dataset was reduced to 2 dimensions using t-SNE.}
   \label{fig:tsne_vis}
\end{figure}

\subsection{Preprocessing}
The images were standardised to prespecified dimensions and colour model and each pixel scaled to a value between 0 and 1. 

Most of the efforts were gone into identifying which colour transformation would be relevant for building better models, and for creating modules for transforming the data to the required format. We aim to explore more preprocessing methods after the mid-evaluation.

\subsection{Data Augmentation}

Modules created for augmentation provide functionality similar to TensorFlow's ImageDataGenerator \cite{tensorflow_imagedatagen}, capable of random rotation, translation, scaling, random noise induction, colour space transformation, cropping, and denoising. We intend to use these to augment the data for model training.

\subsection{Preliminary Models}

Some algorithms were implemented on unaugmented data. The 70\% of data was reserved for training, 15\% for validation and 15\% for testing. Hyperparameters were tuned based on results observed on the validation data and final evaluation done on the testing data. The images were resized to 25x25 dimensions and colour was preserved as RGB. For training, the images were flattened to a 25*25*3 array. Images which were ambiguous \cite{fuhadmalaria} were removed from the dataset, leaving us with 8770 parasitized and 8701 uninfected training images, 2392 parasitized and 2373 unparasitized images for the validation dataset, and 1970 parasitized and 1955 uninfected images for the test dataset. Data augmentation was not used, and we intend to develop it more after the mid-evaluation

\subsubsection{Naive Bayes}
We used Gaussian Naive Bayes without prior weight initialisation.

\subsubsection{Logistic regression}
Logistic Regression was used with the default parameters.

\subsubsection{Decision trees}
Decision trees provide explainable modelling and are very fast to inference with. A decision tree with Gini entropy as the criterion with a maximum depth of 4 was constructed.

\subsubsection{XGBoost}
XGBoost with GPU-acceleration was used, the maximum depth was limited to 5, a forest was created from 20 trees.

\subsubsection{CNNs}
A basic convolutional neural network was implemented using TensorFlow. We intend to explore more on how the layers affect the model training and convergence

\begin{figure}[t]
   \begin{center}
      \includegraphics[width=1\linewidth]{../Plots/model_specifications.png}
   \end{center}
      \caption{CNN model architecture}
   \label{fig:cnn_architecture}
\end{figure}

\subsubsection{Transfer Learning}
Some attempts were taken at transfer learning. We noticed that Xception performed reasonably well, coming close to the CNN, with 95\% validation accuracy. A resizing layer was added to the model which upscaled the dimension of the image to 72x72 from 25x25 to be compatible with Xception.

\section{Results and Analysis}

Below are prelimnary results on unaugmented datasets used for training across 6 different models. Hyperparameters were tuned based on the validation set, and the models tested on the test set. The results can be noted in tables 1, 2, 3 and 4, which correspond to accuracy, precision, recall and f1-score respectively. 


\begin{table}[]
   \begin{tabular}{|llll|}
   \hline
   \multicolumn{4}{|c|}{{\ul \textbf{Accuracy}}}                                                                                  \\ \hline
   \multicolumn{1}{|l|}{Model}                        & \multicolumn{1}{l|}{Training} & \multicolumn{1}{l|}{Validation} & Testing \\ \hline
   \multicolumn{1}{|l|}{Naive Bayes}                  & \multicolumn{1}{l|}{0.639}    & \multicolumn{1}{l|}{0.640}      & 0.632   \\ \hline
   \multicolumn{1}{|l|}{Log. Reg.}          & \multicolumn{1}{l|}{0.750}    & \multicolumn{1}{l|}{0.709}      & 0.707   \\ \hline
   \multicolumn{1}{|l|}{Decision Trees}               & \multicolumn{1}{l|}{0.704}    & \multicolumn{1}{l|}{0.702}      & 0.694   \\ \hline
   \multicolumn{1}{|l|}{XGBoost}                      & \multicolumn{1}{l|}{0.974}    & \multicolumn{1}{l|}{0.865}      & 0.856   \\ \hline
   \multicolumn{1}{|l|}{CNN}                          & \multicolumn{1}{l|}{0.991}    & \multicolumn{1}{l|}{0.991}      & 0.988   \\ \hline
   \multicolumn{1}{|l|}{Trans. Learn.} & \multicolumn{1}{l|}{0.970}    & \multicolumn{1}{l|}{0.948}      & 0.953   \\ \hline
   \end{tabular}

   \caption{Accuracy}
\end{table}

\begin{table}[]
   \begin{tabular}{|llll|}
   \hline
   \multicolumn{4}{|c|}{{\ul \textbf{Precision}}}                                                                   \\ \hline
   \multicolumn{1}{|l|}{Model}          & \multicolumn{1}{l|}{Training} & \multicolumn{1}{l|}{Validation} & Testing \\ \hline
   \multicolumn{1}{|l|}{Naive Bayes}    & \multicolumn{1}{l|}{0.685}    & \multicolumn{1}{l|}{0.683}      & 0.677   \\ \hline
   \multicolumn{1}{|l|}{Log. Reg.}      & \multicolumn{1}{l|}{0.763}    & \multicolumn{1}{l|}{0.716}      & 0.720   \\ \hline
   \multicolumn{1}{|l|}{Decision Trees} & \multicolumn{1}{l|}{0.740}    & \multicolumn{1}{l|}{0.737}      & 0.726   \\ \hline
   \multicolumn{1}{|l|}{XGBoost}        & \multicolumn{1}{l|}{0.985}    & \multicolumn{1}{l|}{0.878}      & 0.869   \\ \hline
   \multicolumn{1}{|l|}{CNN}            & \multicolumn{1}{l|}{0.996}    & \multicolumn{1}{l|}{0.995}      & 0.996   \\ \hline
   \multicolumn{1}{|l|}{Trans. Learn.}  & \multicolumn{1}{l|}{0.971}    & \multicolumn{1}{l|}{0.946}      & 0.955   \\ \hline
   \end{tabular}

\caption{Precision}
\end{table}

% Please add the following required packages to your document preamble:
% \usepackage[normalem]{ulem}
% \useunder{\uline}{\ul}{}
\begin{table}[]
   \begin{tabular}{|llll|}
   \hline
   \multicolumn{4}{|c|}{{\ul \textbf{Recall}}}                                                                      \\ \hline
   \multicolumn{1}{|l|}{Model}          & \multicolumn{1}{l|}{Training} & \multicolumn{1}{l|}{Validation} & Testing \\ \hline
   \multicolumn{1}{|l|}{Naive Bayes}    & \multicolumn{1}{l|}{0.512}    & \multicolumn{1}{l|}{0.529}      & 0.511   \\ \hline
   \multicolumn{1}{|l|}{Log. Reg.}      & \multicolumn{1}{l|}{0.730}    & \multicolumn{1}{l|}{0.698}      & 0.679   \\ \hline
   \multicolumn{1}{|l|}{Decision Trees} & \multicolumn{1}{l|}{0.634}    & \multicolumn{1}{l|}{0.639}      & 0.679   \\ \hline
   \multicolumn{1}{|l|}{XGBoost}        & \multicolumn{1}{l|}{0.962}    & \multicolumn{1}{l|}{0.850}      & 0.839   \\ \hline
   \multicolumn{1}{|l|}{CNN}            & \multicolumn{1}{l|}{0.986}    & \multicolumn{1}{l|}{0.986}      & 0.980   \\ \hline
   \multicolumn{1}{|l|}{Trans. Learn.}  & \multicolumn{1}{l|}{0.970}    & \multicolumn{1}{l|}{0.952}      & 0.952   \\ \hline
   \end{tabular}

   \caption{Recall}
\end{table}



\begin{table}[]
   \begin{tabular}{|llll|}
   \hline
   \multicolumn{4}{|c|}{{\ul \textbf{F1-Score}}}                                                                    \\ \hline
   \multicolumn{1}{|l|}{Model}          & \multicolumn{1}{l|}{Training} & \multicolumn{1}{l|}{Validation} & Testing \\ \hline
   \multicolumn{1}{|l|}{Naive Bayes}    & \multicolumn{1}{l|}{0.591}    & \multicolumn{1}{l|}{0.596}      & 0.583   \\ \hline
   \multicolumn{1}{|l|}{Log. Reg.}      & \multicolumn{1}{l|}{0.746}    & \multicolumn{1}{l|}{0.707}      & 0.699   \\ \hline
   \multicolumn{1}{|l|}{Decision Trees} & \multicolumn{1}{l|}{0.683}    & \multicolumn{1}{l|}{0.679}      & 0.674   \\ \hline
   \multicolumn{1}{|l|}{XGBoost}        & \multicolumn{1}{l|}{0.973}    & \multicolumn{1}{l|}{0.863}      & 0.854   \\ \hline
   \multicolumn{1}{|l|}{CNN}            & \multicolumn{1}{l|}{0.991}    & \multicolumn{1}{l|}{0.990}      & 0.988   \\ \hline
   \multicolumn{1}{|l|}{Trans. Learn.}  & \multicolumn{1}{l|}{0.971}    & \multicolumn{1}{l|}{0.949}      & 0.953   \\ \hline
   \end{tabular}
   \caption{F1-score}
\end{table}

\section{Conclusion}

We notice that the CNN model has the best performance when it comes to accuracy, precision, recall and f1-score and shows low bias and low variance. Meanwhile, the XGBoost model performs extremely well on the training set, but fails to perform as well on the validation and test sets, indicating high variance. The rest of the models: Naive Bayes, Logistic Regression and Decision trees portray high bias and low variance. 

Image space transformations can accentuate features and we intend to explore more regarding that.


\subsection{Remaining Tasks}

We intend to evaluate models on images of varying sizes, from 25x25 to 100x100. We also aim to apply dataset augmentation techniques and nonlinear colour space transformations (eg. RGB to HSV) and determine their impact, while tuning the models further, implementing more architectures and determining optimal hyperparameters. We will explore rule-based thresholding for identifying clustered points. Finally, we will attempt wholeslide image segmentation to isolate RBCs, if time permits.


\subsection{Learnings}

We learnt the importance of literature reviews, which informed us of the latest developments and about the discrepancy in labelling of the dataset \cite{fuhadmalaria}. Non-linear image space transformations can allow for better feature extraction. We also learnt about working with image datasets, what goes into their preprocessing, how to feed them as data to train models.

\subsection{Individual Contributions}

Individual contributions can be found in table 5.
% Please add the following required packages to your document preamble:
% \usepackage[normalem]{ulem}
% \useunder{\uline}{\ul}{}
\begin{table}[]
   \begin{tabular}{ll}
   \multicolumn{2}{c}{{\ul \textbf{Individual contribution}}}                        \\ \hline
   \multicolumn{1}{|l|}{{\ul Name}} & \multicolumn{1}{l|}{{\ul Contribution}}        \\ \hline
   \multicolumn{1}{|l|}{Harshit}    & \multicolumn{1}{l|}{EDA, Testing mod.}         \\ \hline
   \multicolumn{1}{|l|}{Madhava}    & \multicolumn{1}{l|}{EDA, Modelling, data mod.} \\ \hline
   \multicolumn{1}{|l|}{Shreya}     & \multicolumn{1}{l|}{Lit. review, Data aug.}    \\ \hline
   \multicolumn{1}{|l|}{Srishti}    & \multicolumn{1}{l|}{Data aug. , lit. review}   \\ \hline
   \end{tabular}

   \caption{Individual Contributions}
\end{table}



%-------------------------------------------------------------------------
% \subsection{Language}

% All manuscripts must be in English.

% \subsection{Dual submission}

% Please refer to the author guidelines on the CVPR 2020 web page for a
% discussion of the policy on dual submissions.

% \subsection{Paper length}
% Papers, excluding the references section,
% must be no longer than eight pages in length. The references section
% will not be included in the page count, and there is no limit on the
% length of the references section. For example, a paper of eight pages
% with two pages of references would have a total length of 10 pages.
% {\bf There will be no extra page charges for CVPR 2020.}

% Overlength papers will simply not be reviewed.  This includes papers
% where the margins and formatting are deemed to have been significantly
% altered from those laid down by this style guide.  Note that this
% \LaTeX\ guide already sets figure captions and references in a smaller font.
% The reason such papers will not be reviewed is that there is no provision for
% supervised revisions of manuscripts.  The reviewing process cannot determine
% the suitability of the paper for presentation in eight pages if it is
% reviewed in eleven.  

% %-------------------------------------------------------------------------
% \subsection{The ruler}
% The \LaTeX\ style defines a printed ruler which should be present in the
% version submitted for review.  The ruler is provided in order that
% reviewers may comment on particular lines in the paper without
% circumlocution.  If you are preparing a document using a non-\LaTeX\
% document preparation system, please arrange for an equivalent ruler to
% appear on the final output pages.  The presence or absence of the ruler
% should not change the appearance of any other content on the page.  The
% camera ready copy should not contain a ruler. (\LaTeX\ users may uncomment
% the \verb'\cvprfinalcopy' command in the document preamble.)  Reviewers:
% note that the ruler measurements do not align well with lines in the paper
% --- this turns out to be very difficult to do well when the paper contains
% many figures and equations, and, when done, looks ugly.  Just use fractional
% references (e.g.\ this line is $095.5$), although in most cases one would
% expect that the approximate location will be adequate.

% \subsection{Mathematics}

% Please number all of your sections and displayed equations.  It is
% important for readers to be able to refer to any particular equation.  Just
% because you didn't refer to it in the text doesn't mean some future reader
% might not need to refer to it.  It is cumbersome to have to use
% circumlocutions like ``the equation second from the top of page 3 column
% 1''.  (Note that the ruler will not be present in the final copy, so is not
% an alternative to equation numbers).  All authors will benefit from reading
% Mermin's description of how to write mathematics:
% \url{http://www.pamitc.org/documents/mermin.pdf}.


% \subsection{Blind review}

% Many authors misunderstand the concept of anonymizing for blind
% review.  Blind review does not mean that one must remove
% citations to one's own work---in fact it is often impossible to
% review a paper unless the previous citations are known and
% available.

% Blind review means that you do not use the words ``my'' or ``our''
% when citing previous work.  That is all.  (But see below for
% techreports.)

% Saying ``this builds on the work of Lucy Smith [1]'' does not say
% that you are Lucy Smith; it says that you are building on her
% work.  If you are Smith and Jones, do not say ``as we show in
% [7]'', say ``as Smith and Jones show in [7]'' and at the end of the
% paper, include reference 7 as you would any other cited work.

% An example of a bad paper just asking to be rejected:
% \begin{quote}
% \begin{center}
%     An analysis of the frobnicatable foo filter.
% \end{center}

%    In this paper we present a performance analysis of our
%    previous paper [1], and show it to be inferior to all
%    previously known methods.  Why the previous paper was
%    accepted without this analysis is beyond me.

%    [1] Removed for blind review
% \end{quote}


% An example of an acceptable paper:

% \begin{quote}
% \begin{center}
%      An analysis of the frobnicatable foo filter.
% \end{center}

%    In this paper we present a performance analysis of the
%    paper of Smith \etal [1], and show it to be inferior to
%    all previously known methods.  Why the previous paper
%    was accepted without this analysis is beyond me.

%    [1] Smith, L and Jones, C. ``The frobnicatable foo
%    filter, a fundamental contribution to human knowledge''.
%    Nature 381(12), 1-213.
% \end{quote}

% If you are making a submission to another conference at the same time,
% which covers similar or overlapping material, you may need to refer to that
% submission in order to explain the differences, just as you would if you
% had previously published related work.  In such cases, include the
% anonymized parallel submission~\cite{cdcmalaria} as additional material and
% cite it as
% \begin{quote}
% [1] Authors. ``The frobnicatable foo filter'', F\&G 2014 Submission ID 324,
% Supplied as additional material {\tt fg324.pdf}.
% \end{quote}

% Finally, you may feel you need to tell the reader that more details can be
% found elsewhere, and refer them to a technical report.  For conference
% submissions, the paper must stand on its own, and not {\em require} the
% reviewer to go to a techreport for further details.  Thus, you may say in
% the body of the paper ``further details may be found
% in~\cite{Authors14b}''.  Then submit the techreport as additional material.
% Again, you may not assume the reviewers will read this material.

% Sometimes your paper is about a problem which you tested using a tool which
% is widely known to be restricted to a single institution.  For example,
% let's say it's 1969, you have solved a key problem on the Apollo lander,
% and you believe that the CVPR70 audience would like to hear about your
% solution.  The work is a development of your celebrated 1968 paper entitled
% ``Zero-g frobnication: How being the only people in the world with access to
% the Apollo lander source code makes us a wow at parties'', by Zeus \etal.

% You can handle this paper like any other.  Don't write ``We show how to
% improve our previous work [Anonymous, 1968].  This time we tested the
% algorithm on a lunar lander [name of lander removed for blind review]''.
% That would be silly, and would immediately identify the authors. Instead
% write the following:
% \begin{quotation}
% \noindent
%    We describe a system for zero-g frobnication.  This
%    system is new because it handles the following cases:
%    A, B.  Previous systems [Zeus et al. 1968] didn't
%    handle case B properly.  Ours handles it by including
%    a foo term in the bar integral.

%    ...

%    The proposed system was integrated with the Apollo
%    lunar lander, and went all the way to the moon, don't
%    you know.  It displayed the following behaviours
%    which show how well we solved cases A and B: ...
% \end{quotation}
% As you can see, the above text follows standard scientific convention,
% reads better than the first version, and does not explicitly name you as
% the authors.  A reviewer might think it likely that the new paper was
% written by Zeus \etal, but cannot make any decision based on that guess.
% He or she would have to be sure that no other authors could have been
% contracted to solve problem B.
% \medskip

% \noindent
% FAQ\medskip\\
% {\bf Q:} Are acknowledgements OK?\\
% {\bf A:} No.  Leave them for the final copy.\medskip\\
% {\bf Q:} How do I cite my results reported in open challenges?
% {\bf A:} To conform with the double blind review policy, you can report results of other challenge participants together with your results in your paper. For your results, however, you should not identify yourself and should not mention your participation in the challenge. Instead present your results referring to the method proposed in your paper and draw conclusions based on the experimental comparison to other results.\medskip\\



% % \begin{figure}[t]
% % \begin{center}
% % \fbox{\rule{0pt}{2in} \rule{0.9\linewidth}{0pt}}
% %    %\includegraphics[width=0.8\linewidth]{egfigure.eps}
% % \end{center}
% %    \caption{Example of caption.  It is set in Roman so that mathematics
% %    (always set in Roman: $B \sin A = A \sin B$) may be included without an
% %    ugly clash.}
% % \label{fig:long}
% % \label{fig:onecol}
% % \end{figure}

% \subsection{Miscellaneous}

% \noindent
% Compare the following:\\
% \begin{tabular}{ll}
%  \verb'$conf_a$' &  $conf_a$ \\
%  \verb'$\mathit{conf}_a$' & $\mathit{conf}_a$
% \end{tabular}\\
% See The \TeX book, p165.

% The space after \eg, meaning ``for example'', should not be a
% sentence-ending space. So \eg is correct, {\em e.g.} is not.  The provided
% \verb'\eg' macro takes care of this.

% When citing a multi-author paper, you may save space by using ``et alia'',
% shortened to ``\etal'' (not ``{\em et.\ al.}'' as ``{\em et}'' is a complete word.)
% However, use it only when there are three or more authors.  Thus, the
% following is correct: ``
%    Frobnication has been trendy lately.
%    It was introduced by Alpher~\cite{Alpher02}, and subsequently developed by
%    Alpher and Fotheringham-Smythe~\cite{Alpher03}, and Alpher \etal~\cite{Alpher04}.''

% This is incorrect: ``... subsequently developed by Alpher \etal~\cite{Alpher03} ...''
% because reference~\cite{Alpher03} has just two authors.  If you use the
% \verb'\etal' macro provided, then you need not worry about double periods
% when used at the end of a sentence as in Alpher \etal.

% For this citation style, keep multiple citations in numerical (not
% chronological) order, so prefer \cite{Alpher03,Alpher02,cdcmalaria} to
% \cite{Alpher02,Alpher03,cdcmalaria}.


% % \begin{figure*}
% % \begin{center}
% % \fbox{\rule{0pt}{2in} \rule{.9\linewidth}{0pt}}
% % \end{center}
% %    \caption{Example of a short caption, which should be centered.}
% % \label{fig:short}
% % \end{figure*}

% %------------------------------------------------------------------------
% \section{Formatting your paper}

% All text must be in a two-column format. The total allowable width of the
% text area is $6\frac78$ inches (17.5 cm) wide by $8\frac78$ inches (22.54
% cm) high. Columns are to be $3\frac14$ inches (8.25 cm) wide, with a
% $\frac{5}{16}$ inch (0.8 cm) space between them. The main title (on the
% first page) should begin 1.0 inch (2.54 cm) from the top edge of the
% page. The second and following pages should begin 1.0 inch (2.54 cm) from
% the top edge. On all pages, the bottom margin should be 1-1/8 inches (2.86
% cm) from the bottom edge of the page for $8.5 \times 11$-inch paper; for A4
% paper, approximately 1-5/8 inches (4.13 cm) from the bottom edge of the
% page.

% %-------------------------------------------------------------------------
% \subsection{Margins and page numbering}

% All printed material, including text, illustrations, and charts, must be kept
% within a print area 6-7/8 inches (17.5 cm) wide by 8-7/8 inches (22.54 cm)
% high.
% Page numbers should be in footer with page numbers, centered and .75
% inches from the bottom of the page and make it start at the correct page
% number rather than the 4321 in the example.  To do this fine the line (around
% line 23)
% \begin{verbatim}
% %\ifcvprfinal\pagestyle{empty}\fi
% \setcounter{page}{4321}
% \end{verbatim}
% where the number 4321 is your assigned starting page.

% Make sure the first page is numbered by commenting out the first page being
% empty on line 46
% \begin{verbatim}
% %\thispagestyle{empty}
% \end{verbatim}


% %-------------------------------------------------------------------------
% \subsection{Type-style and fonts}

% Wherever Times is specified, Times Roman may also be used. If neither is
% available on your word processor, please use the font closest in
% appearance to Times to which you have access.

% MAIN TITLE. Center the title 1-3/8 inches (3.49 cm) from the top edge of
% the first page. The title should be in Times 14-point, boldface type.
% Capitalize the first letter of nouns, pronouns, verbs, adjectives, and
% adverbs; do not capitalize articles, coordinate conjunctions, or
% prepositions (unless the title begins with such a word). Leave two blank
% lines after the title.

% AUTHOR NAME(s) and AFFILIATION(s) are to be centered beneath the title
% and printed in Times 12-point, non-boldface type. This information is to
% be followed by two blank lines.

% The ABSTRACT and MAIN TEXT are to be in a two-column format.

% MAIN TEXT. Type main text in 10-point Times, single-spaced. Do NOT use
% double-spacing. All paragraphs should be indented 1 pica (approx. 1/6
% inch or 0.422 cm). Make sure your text is fully justified---that is,
% flush left and flush right. Please do not place any additional blank
% lines between paragraphs.

% Figure and table captions should be 9-point Roman type as in
% Figures~\ref{fig:onecol} and~\ref{fig:short}.  Short captions should be centred.

% \noindent Callouts should be 9-point Helvetica, non-boldface type.
% Initially capitalize only the first word of section titles and first-,
% second-, and third-order headings.

% FIRST-ORDER HEADINGS. (For example, {\large \bf 1. Introduction})
% should be Times 12-point boldface, initially capitalized, flush left,
% with one blank line before, and one blank line after.

% SECOND-ORDER HEADINGS. (For example, { \bf 1.1. Database elements})
% should be Times 11-point boldface, initially capitalized, flush left,
% with one blank line before, and one after. If you require a third-order
% heading (we discourage it), use 10-point Times, boldface, initially
% capitalized, flush left, preceded by one blank line, followed by a period
% and your text on the same line.

% %-------------------------------------------------------------------------
% \subsection{Footnotes}

% Please use footnotes\footnote {This is what a footnote looks like.  It
% often distracts the reader from the main flow of the argument.} sparingly.
% Indeed, try to avoid footnotes altogether and include necessary peripheral
% observations in
% the text (within parentheses, if you prefer, as in this sentence).  If you
% wish to use a footnote, place it at the bottom of the column on the page on
% which it is referenced. Use Times 8-point type, single-spaced.


% %-------------------------------------------------------------------------
% \subsection{References}

% List and number all bibliographical references in 9-point Times,
% single-spaced, at the end of your paper. When referenced in the text,
% enclose the citation number in square brackets, for
% example~\cite{cdcmalaria}.  Where appropriate, include the name(s) of
% editors of referenced books.

% \begin{table}
% \begin{center}
% \begin{tabular}{|l|c|}
% \hline
% Method & Frobnability \\
% \hline\hline
% Theirs & Frumpy \\
% Yours & Frobbly \\
% Ours & Makes one's heart Frob\\
% \hline
% \end{tabular}
% \end{center}
% \caption{Results.   Ours is better.}
% \end{table}

% %-------------------------------------------------------------------------
% \subsection{Illustrations, graphs, and photographs}

% All graphics should be centered.  Please ensure that any point you wish to
% make is resolvable in a printed copy of the paper.  Resize fonts in figures
% to match the font in the body text, and choose line widths which render
% effectively in print.  Many readers (and reviewers), even of an electronic
% copy, will choose to print your paper in order to read it.  You cannot
% insist that they do otherwise, and therefore must not assume that they can
% zoom in to see tiny details on a graphic.

% When placing figures in \LaTeX, it's almost always best to use
% \verb+\includegraphics+, and to specify the  figure width as a multiple of
% the line width as in the example below
% {\small\begin{verbatim}
%    \usepackage[dvips]{graphicx} ...
%    \includegraphics[width=0.8\linewidth]
%                    {myfile.eps}
% \end{verbatim}
% }


% %-------------------------------------------------------------------------
% \subsection{Color}

% Please refer to the author guidelines on the CVPR 2020 web page for a discussion
% of the use of color in your document.

% %------------------------------------------------------------------------
% \section{Final copy}

% You must include your signed IEEE copyright release form when you submit
% your finished paper. We MUST have this form before your paper can be
% published in the proceedings.

% Please direct any questions to the production editor in charge of these 
% proceedings at the IEEE Computer Society Press: 
% \url{https://www.computer.org/about/contact}. 


{\small
\bibliographystyle{ieeetr}
\bibliography{egbib}
}

\end{document}
